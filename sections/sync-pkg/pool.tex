\subsection{池\Pool }
\begin{frame}{池\Pool }
    \begin{itemize}
        \item \textbf{用途}: \Pool 创建并导出特定数目或一批资源以被第三方使用
        \item 3个基本方法
            \begin{description}
                \item[New] 创建资源实例
                \item[Get] 调用时,如果池有可用的实例,则将其直接返回给请求方,否则调用New方法实时创建一个,然后将其返回给请求方
                \item[Put] 将服务完毕的实例放回池缓存以被其他进程使用
            \end{description}
    \end{itemize}
\end{frame}

\begin{frame}[fragile]{实例}
    利用池创建对象,\alert{使用完毕后重新放回池中},从而节省对象的创建
\begin{lstlisting}
myPool := &sync.Pool{
  New: func() interface{} {
    fmt.Println("Creating new instance.")
    return struct{}{}
  },
}
myPool.Get()
instance := myPool.Get()
myPool.Put(instance)
myPool.Get()

// 输出:
// Creating new instance.
// Creating new instance.
\end{lstlisting}

另一个例子是\mhref{https://github.com/sammyne/concurrency-in-go/blob/master/chapter03/sync.pkg/pool/basic2.go}{使用池缓存资源以节省内存空间}

\end{frame}

\begin{frame}{更多使用场景}
    预存一批能够快速响应(初始化比较漫长)的服务提供方,相关代码参见
    \begin{itemize}
        \item \mhref{https://github.com/sammyne/concurrency-in-go/blob/master/chapter03/sync.pkg/pool/slow_network_service_benchmark_test.go}{没有使用池预存对象的方式}
        \item \mhref{https://github.com/sammyne/concurrency-in-go/blob/master/chapter03/sync.pkg/pool/slow_network_service_benchmark_test.go}{使用池预存对象的方式}
    \end{itemize}

    \bigskip
    对比结果是:使用缓存池的方式快了近 1000 倍
\end{frame}

\begin{frame}{注意事项}
\begin{exampleblock}{适用条件}
    \begin{itemize}
        \item 为并发的进程提供服务对象,这些对象需要满足
            \begin{itemize}
                \item 在初始化后很快就被抛弃
                \item 或创建会对内存影响不小
            \end{itemize}
    \end{itemize}
\end{exampleblock}    

\pause
\begin{alertblock}{不适用条件}
   所需服务对象不是同态的---部分时间会被浪费在对象的类型转换上 
\end{alertblock}
\end{frame}

\begin{frame}{\Pool 小结}
    \begin{itemize}
        \item 初始化\Pool 时,创建一个\code{New}成员
        \item 不要对\code{Get()}返回的对象状态做任何前提假设
        \item 不要忘记利用\code{Put()}将服务对象再次缓存以重用
        \item 池中对象应基本同态,即类型相同    
    \end{itemize}
\end{frame}