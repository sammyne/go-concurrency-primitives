\subsection{条件信号量\Cond}
\begin{frame}{条件信号量\Cond}
    \begin{itemize}
        \item \textbf{用途}: 协程用于等待或通知特定事件发生的消息中心
        \item 这里的事件只在两个或更多协程之间传递事件发生的事实本身,\alert{无法承载其他信息}
    \end{itemize}
\end{frame}

\begin{frame}[fragile]{使用场景}
   没有借助条件信号量实现事件通知的两种方式
   
   \begin{enumerate}
       \item \code{for}循环等到为止: 无限占用 CPU 内核,\alert{无法被调度}
\begin{lstlisting}
for conditionTrue() == false { }    
\end{lstlisting}
       \item 利用主动睡眠提供可被抢占机会: \alert{效率不高,且难以准确设置睡眠的时长}:太长降低性能,太短消耗太多 CPU 时间    
\begin{lstlisting}
for conditionTrue() == false {
  time. Sleep(1*time.Millisecond)
}
\end{lstlisting}
   \end{enumerate}
\end{frame}

\begin{frame}[fragile]{使用场景}
   借助\Cond 实现事件通知的更优方式
\begin{lstlisting}
c := sync.NewCond(&sync.Mutex{})

c.L.Lock()
for conditionTrue() == false {
  c.Wait()
}
c.L.Unlock()
\end{lstlisting}

\begin{itemize}
    \item \code{Wait()}促使协程进入阻塞状态,挂起当前协程,为其他协程腾出在系统级线程执行的机会
    \item \code{Wait()}内部执行过程会先调用所绑定锁的\code{Unlock()},而在收到需要的信号后、退出执行前再次执行\code{Lock()}重新获得对锁的控制权
\end{itemize}

另一个例子是\mhref{https://github.com/sammyne/concurrency-in-go/blob/master/chapter03/sync.pkg/cond/queue.go}{玩具级的限定容量队列}
\end{frame}

\begin{frame}{两种通知方式: \code{Signal}和\code{Broadcast}}
   \begin{itemize}
       \item Go 运行时维护着等待信号的协程队列(先进先出)
       \item \code{Signal()}调用后,Go 运行时只会通知等待最久的那个协程并将其出列
       \item \code{Broadcast()}的调用则会(逐一?并发?)通知所有等待的协程
       \item \Cond 的性能要比\channel 的高
   \end{itemize} 

   某些需求场景下,基于\Cond 的实现会比基于通道\channel 简洁很多。例如,\mhref{https://github.com/sammyne/concurrency-in-go/blob/master/chapter03/sync.pkg/cond/boardcast.go}{按钮点击事件的通知}
\end{frame}