\section{\select 语句}
\begin{frame}[fragile]{\select 语句}
    \begin{itemize}
        \item \select 语句负责绑定通道到
            \begin{itemize}
                \item 局部的类型或单个函数
                \item 全局范围内多个系统组件
            \end{itemize}
        \item 借助\select 语句安全地利用通道实现诸如\alert{取消}、\alert{超时}(\lstlistingname~\ref{lst-select-based-timeout})、\alert{等待}和\alert{默认值}等概念
        \item 空\select 语句(\code{select \{\}})等价于一个死循环,会永远阻塞
    \end{itemize}

\begin{lstlisting}[caption={基于\select 实现超时退出},label=lst-select-based-timeout]
var c <-chan int
select {
case <-c:
case <-time.After(1 * time.Second):
    fmt.Println("Timed out.")
}

// 输出:
// Timed out.
\end{lstlisting}
    
\end{frame}

\begin{frame}{\texttt{case}子句的选择公平性}
    \begin{itemize}
        \item 不同于\code{switch}语句,对于\select 语句
            \begin{itemize}
                \item \code{case}子句不会线性地从上往下执行
                \item 在没有任何满足条件的\code{case}子块时会一直阻塞
                \item Go 运行时会随机均匀地选择非阻塞的\code{case}子句执行
            \end{itemize}
    \end{itemize}

\code{case}语句指明的所有通道读写操作会被同时检查是否已被满足,即
\begin{itemize}
    \item 读操作的可读通道非空,或通道已关闭 
    \item 写操作的可写通道非满
\end{itemize}

\end{frame}

\begin{frame}[fragile]{\texttt{case}子句的选择公平性证明}
\begin{lstlisting}
c1 := make(chan interface{})
close(c1)
c2 := make(chan interface{})
close(c2)

var c1Count, c2Count int
for i := 1000; i >= 0; i-- {
  select {
  case <-c1:
    c1Count++
  case <-c2:
    c2Count++
  }
}
fmt.Printf("c1Count: %d\nc2Count: %d\n", c1Count, c2Count)

// 输出
// c1Count: 495
// c2Count: 506
\end{lstlisting}    
\end{frame}

\begin{frame}[fragile]{\texttt{default}默认子句}
\begin{itemize}
    \item \textbf{用途}: \code{default}子句负责执行其他\code{case}子句都不满足时的操作
\begin{lstlisting}
start := time.Now()
var c1, c2 <-chan int

select {
case <-c1:
case <-c2:
default:
  fmt.Println("In default after", time.Since(start))
}

// 输出:
// In default after 1.482µs
\end{lstlisting}
    \item 通常\code{default}子句会与\code{for-select}循环结合使用。其中一个\href{https://github.com/sammyne/concurrency-in-go/blob/master/chapter03/select/for_and_default.go}{用例}为:利用一个通道\code{done}作为终止循环的信号,而\code{default}子句负责执行终止之前的常规任务
\end{itemize}
\end{frame}