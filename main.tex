%!TEX program=xelatex
\documentclass{beamer}
\usetheme{focus}
% Full instructions available at:
% https://github.com/elauksap/focus-beamertheme

\usepackage{ctex}

\usepackage{listings}
\usepackage{golang}
\usepackage{xcolor}

\title{Go并发元件}
%\subtitle{Subtitle}
\author{黎相敏}
\titlegraphic{\includegraphics[scale=1.25]{focuslogo.pdf}}
\institute{上海观源信息科技有限公司 \\ 上海市闵行区紫竹科技园4号楼303B}
\date{12/01/2019}

\lstset{ % add your own preferences
    frame=trBL, %single,
    basicstyle=\ttfamily\footnotesize,
    keywordstyle=\color{red},
    numbers=left,
    numbersep=8pt,
    showstringspaces=false, 
    stringstyle=\color{blue},
    tabsize=2,
    language=golang % this is it !
}
\renewcommand{\lstlistingname}{代码片段}% Listing -> 代码片段

% command macros to save typing
\newcommand{\channel}{\texttt{channel}}
\newcommand{\Cond}{\texttt{Cond}}
\newcommand{\Mutex}{\texttt{Mutex}}
\newcommand{\Once}{\texttt{Once}}
\newcommand{\Pool}{\texttt{Pool}}
\newcommand{\RWMutex}{\texttt{RWMutex}}
\newcommand{\WaitGroup}{\texttt{WaitGroup}}

\begin{document}

    \begin{frame}
        \maketitle
    \end{frame}

    \begin{frame}{大纲}
        \tableofcontents
    \end{frame}

    %\section{基本术语}
\begin{frame}{分叉--汇合(Fork-Join)模型}
  \begin{columns}
      \column{.5\textwidth}
        \begin{itemize}
          \item 程序执行过程中,\alert{父线程}可以分叉出与其并发执行的\alert{子线程}
          \item \alert{汇合点}: 子线程从独立执行到汇合回父线程的时间点
        \end{itemize}

      \column{.5\textwidth}
      \begin{figure}
        \includegraphics[width=\textwidth]{images/fork-join.png}
        \caption{分叉--汇合模型}
      \end{figure}
  \end{columns}
\end{frame}

\begin{frame}[fragile]{错误设置的汇合点}
\begin{lstlisting}[caption={不正确的汇合点引发竞态},label={wrong-join}]
sayHello := func() {
  fmt.Println("hello")
}
go sayHello()
// continue doing other things
\end{lstlisting}

  \alert{以上程序的输出结果是不确定的}。因为协程由 Go 运行时创建和调度,\code{sayHello}协程没能通过合适的汇合点和主协程进行汇合。因此,如果被调度之前主协程已退出,\code{sayHello}就无法获得执行的机会

  \pause
  \begin{exampleblock}{温馨提示}
      \textbf{正确设置的汇点}能确保程序正确性并消除潜在的\alert{竞态} 
  \end{exampleblock}
\end{frame}

\begin{frame}[fragile]{纠正的汇合点}
为上述程序设置正确的汇合点,\texttt{sayHello}必须和主协程\textbf{同步},使自己能在主协程退出之前与其汇合,解决方案之一如\lstlistingname~\ref{lst-ok-join}

\begin{lstlisting}[caption={利用同步确保\texttt{sayHello}在主协程退出之前与其汇合},label=lst-ok-join]
var wg sync. WaitGroup
sayHello := func() {
  defer wg.Done()
  fmt.Println("hello")
}
wg.Add(1)
go sayHello()
wg.Wait()  // 汇合点

// 输出:
// hello    
\end{lstlisting}

\alert{可见,分叉---汇合模型下并发编程的正确性依赖于数据同步}
\end{frame}

\begin{frame}{数据同步方式}
\texttt{go}语言实现数据同步操作提供了两种方式
\begin{itemize}
    \item 传统地,\alert{共享内存型同步模式},常用元件分布在\texttt{sync}包
    \item \alert{基于顺序进程通信(CSP)}的消息传递实现数据同步,主要元件为通道\channel 及\select 语句
\end{itemize}
\end{frame}

    \section{sync包}
\begin{frame}{基本元件}
   \texttt{sync}包维护着用于同步底层内存访问的元件,包括
   \begin{itemize}
       \item \WaitGroup
       \item \Mutex 和 \RWMutex
       \item \Cond
       \item \Once
       \item \Pool
   \end{itemize} 
\end{frame}

    %\begin{frame}[fragile]{\WaitGroup}
   \textbf{用途}: 等待一批并发操作结束,\alert{操作的结果不是关心的重点或者能够通过其他途径收集} 

   \begin{columns}[t]
       \column{0.5\textwidth}
\begin{lstlisting}[xleftmargin=8pt]
var wg sync.WaitGroup

wg.Add(1)
go func() {
  defer wg.Done()
  fmt.Println("Alice sleeping...")
  time.Sleep(1)
}()

wg.Add(1)
go func() {
  defer wg.Done()
  fmt.Println("Bob sleeping...")
\end{lstlisting}

       \column{0.5\textwidth}
\begin{lstlisting}[firstnumber=last,xleftmargin=16pt]
  time.Sleep(2)
}()
wg.Wait()
fmt.Println("Both are awaken.")

// 输出:
// Bob sleeping...
// Alice sleeping...
// Both are awaken.
\end{lstlisting}
   \end{columns}
\end{frame}

\begin{frame}{\WaitGroup}
    \WaitGroup 可看作一个线程安全的计数器

    \begin{itemize}
        \item 通过 \texttt{Add(x)}加上特定数\texttt{x}
        \item 通过 \texttt{Done()}使其减1
        \item \texttt{Wait()}使其一直阻塞直至计数器置零
    \end{itemize}

    \begin{alertblock}{温馨提示}
        \texttt{Add()}需要在\WaitGroup 协调的协程之外(一般使主协程)调用,否则将引入\alert{竞态}
    \end{alertblock}
\end{frame}
    %\subsection{通用锁\Mutex 和读写锁\RWMutex }
\begin{frame}{通用锁\Mutex 和读写锁\RWMutex }
    \begin{itemize}
        \item \Mutex 是``Mutual Exclusion''的缩写
        \item \textbf{用途}: 保护程序的\alert{关键区域}---需要排他地存取共享资源的程序片段
        \item \Mutex 和\RWMutex 要求开发者必须以\alert{特定的方式}访问内存以保证数据的同步
    \end{itemize}

    \mhref{https://github.com/sammyne/concurrency-in-go/blob/master/chapter03/sync.pkg/mutex/basic.go}{一个利用锁保证计数器安全读写的例子}

    \pause
    \begin{exampleblock}{温馨提示}
        \begin{itemize}
            \item 锁\code{m}的\code{Lock()}调用之后应有配对的\code{defer m.UnLock()}语句
            \item \alert{进出关键区域的代价是昂贵的}
        \end{itemize}    
    \end{exampleblock}
\end{frame}

\iffalse
\begin{frame}[fragile]{示例}
\begin{lstlisting}
var count int
var lock sync.Mutex

increment := func() {
  lock.Lock()
  defer lock.Unlock()
  count++
  fmt.Printf("Incrementing: %d\n", count)
}
decrement := func() {
  lock.Lock() // 请求对关键区域--count变量的存取
  defer lock.Unlock()  // 放弃对关键区域的排他存取权利
  count--
  fmt.Printf("Decrementing: %d\n", count)
}
// Increment
var arithmetic sync.WaitGroup
for i := 0; i <= 2; i++ {
  arithmetic.Add(1)
  go func() {
    defer arithmetic.Done()
    increment()
  }()
}
\end{lstlisting}
\end{frame}

\begin{frame}[fragile]{示例(续)}
    \begin{columns}[t]
        \begin{column}{0.5\textwidth}
\begin{lstlisting}[caption={\Mutex 使用示例(续)},firstnumber=last,xleftmargin=8pt]
// Decrement
for i := 0; i <= 2; i++ {
  arithmetic.Add(1)
  go func() {
    defer arithmetic.Done()
\end{lstlisting}
        \end{column}
        \begin{column}{0.5\textwidth}
\begin{lstlisting}[caption={\Mutex 使用示例(续)},firstnumber=last,xleftmargin=8pt]
    decrement()
  }()
}
arithmetic.Wait()
fmt.Println("Arithmetic complete.")

// 输出:
// Decrementing: -1
// Incrementing: 0
// Incrementing: 1
// Decrementing: 0
// Decrementing: -1
// Incrementing: 0
// Arithmetic complete.
\end{lstlisting}
        \end{column}
    \end{columns}
\end{frame}

\begin{frame}{温馨提示}
    \begin{itemize}
        \item 锁\texttt{m}的\texttt{Lock()}调用之后应有配对的\texttt{defer m.UnLock()}语句
        \item \alert{进出关键区域的代价是昂贵的}
    \end{itemize}    
\end{frame}
\fi

\begin{frame}{读写锁\RWMutex }
    \alert{在写锁未被锁定之前},读写锁能够满足任意共存的读锁请求

    \bigskip
    示例代码参见\href{https://github.com/sammyne/concurrency-in-go/blob/master/chapter03/sync.pkg/mutex/rwlock.go}{\Mutex 和\RWMutex 性能对比},具体结果如下图所示,

    \begin{figure}
        \centering
        \includegraphics[width=0.7\textwidth]{images/rwmutex-vs-mutex.png}
        \caption{\Mutex 和\RWMutex 性能对比结果}
    \end{figure}
    
    \alert{可见,在给定的数据范围内,性能提升并不明显}
\end{frame}

\iffalse
\begin{frame}{读写锁\RWMutex }
    \alert{在写锁未被锁定之前},读写锁能够满足任意共存的读锁请求

    示例代码参见\href{https://github.com/sammyne/concurrency-in-go/blob/master/chapter03/sync.pkg/mutex/rwlock.go}{\Mutex 和\RWMutex 性能对比}

    程序的输出结果类似
    \begin{columns}
        \column{0.5\textwidth}
            \begin{table}
                \centering
                \caption{\Mutex 和\RWMutex 性能对比}
                \begin{tabular}{lll}
                    \hline
                    Readers  &RWMutex       &Mutex  \\
                    \hline
                    1        &32.87µs       &5.541µs \\
                    2        &19.603µs      &5.439µs \\
                    4        &76.62µs       &15.886µs \\
                    8        &20.201µs      &26.892µs \\
                    16       &51.558µs      &50.657µs \\
                    32       &61.313µs      &34.999µs \\
                    \hline
                \end{tabular}
            \end{table}

        \column{0.5\textwidth}
            \begin{table}
                \centering
                \caption{\Mutex 和\RWMutex 性能对比(续)}
                \begin{tabular}{lll}
                    \hline
                    Readers  &RWMutex       &Mutex  \\
                    \hline
                    64       &79.628µs      &54.763µs \\
                    128      &96.749µs      &118.701µs \\
                    256      &75.414µs      &89.375µs \\
                    512      &142.882µs     &114.705µs \\
                    1024     &239.471µs     &289.861µs \\
                    2048     &540.809µs     &479.173µs \\ 
                    \hline
                \end{tabular}
            \end{table}
    \end{columns}
\end{frame}

\begin{frame}{读写锁\RWMutex }
            \begin{table}[htbp!]
                \caption{\Mutex 和\RWMutex 性能对比(续)}
                \begin{tabular}{lll}
                    \hline
                    Readers  &RWMutex       &Mutex  \\
                    \hline
                    2048     &540.809µs     &479.173µs \\ 
                    4096     &4.982512ms    &827.095µs \\
                    8192     &2.09599ms     &1.790277ms \\
                    16384    &4.47045ms     &3.820926ms \\
                    32768    &7.911863ms    &7.163938ms \\
                    65536    &15.689641ms   &14.66057ms \\
                    131072   &31.016011ms   &28.674835ms \\
                    262144   &62.493129ms   &56.609731ms \\
                    524288   &121.927969ms  &113.786247ms \\   
                    \hline
                \end{tabular}
            \end{table} 
\end{frame}
\fi
    %\subsection{条件信号量\Cond}
\begin{frame}{条件信号量\Cond}
    \begin{itemize}
        \item \textbf{用途}: 协程用于等待或通知特定事件发生的\alert{消息中心}
        \item 这里的事件只在两个或更多协程之间传递事件发生的事实本身,\alert{无法承载其他信息}
    \end{itemize}
\end{frame}

\begin{frame}[fragile]{使用场景}
   不借助条件信号量实现事件通知的两种方式
   
   \begin{enumerate}
       \item \code{for}循环等到为止: 无限占用 CPU 内核,\alert{无法被调度}
\begin{lstlisting}
for conditionTrue() == false { }    
\end{lstlisting}
       \item 利用主动睡眠提供可被抢占机会: \alert{效率不高,且难以准确设置睡眠的时长}:太长降低性能,太短消耗太多 CPU 时间    
\begin{lstlisting}
for conditionTrue() == false {
  time. Sleep(1*time.Millisecond)
}
\end{lstlisting}
   \end{enumerate}
\end{frame}

\begin{frame}[fragile]{使用场景}
   借助\Cond 实现事件通知的更优方式
\begin{lstlisting}
c := sync.NewCond(&sync.Mutex{})

c.L.Lock()
for conditionTrue() == false {
  c.Wait()
}
c.L.Unlock()
\end{lstlisting}

\begin{itemize}
    \item \code{Wait()}促使协程进入阻塞状态,挂起当前协程,为其他协程腾出在系统级线程执行的机会
    \item \code{Wait()}内部执行过程会先调用所绑定锁的\code{Unlock()},而在收到需要的信号后、退出执行前再次执行\code{Lock()}重新获得锁的控制权
\end{itemize}

另一个例子是\mhref{https://github.com/sammyne/concurrency-in-go/blob/master/chapter03/sync.pkg/cond/queue.go}{玩具级的限定容量队列}
\end{frame}

\begin{frame}{两种通知方式: \code{Signal}和\code{Broadcast}}
   \begin{itemize}
       \item Go 运行时维护着等待信号的协程队列(先进先出)
       \item \code{Signal()}调用后,Go 运行时只会通知等待最久的那个协程并将其出列
       \item \code{Broadcast()}的调用则会(逐一?并发?)通知所有等待的协程
       \item \Cond 的性能要比\channel 的高
   \end{itemize} 

   某些需求场景下,基于\Cond 的实现会比基于通道\channel 简洁很多。例如,\mhref{https://github.com/sammyne/concurrency-in-go/blob/master/chapter03/sync.pkg/cond/boardcast.go}{按钮点击事件的通知}
\end{frame}
    %\subsection{\Once }
\begin{frame}[fragile]{\Once }
  \text{用途}: 确保某个操作最多只被执行一次

\begin{columns}[T]
    \column{.45\textwidth}    
\begin{lstlisting}[caption={\Once 使用样例},xleftmargin=8pt]
var count int
increment := func() {
  count++
}

var once sync.Once
var increments sync.WaitGroup

increments.Add(100)
for i := 0; i < 100; i++ {
  go func() {
    defer increments.Done()
    once.Do(increment)
  }()
}
\end{lstlisting}

    \column{.5\textwidth}    
\begin{lstlisting}[caption={\Once 使用样例},firstnumber=last,xleftmargin=8pt]
increments.Wait()
fmt.Println("Count is", count)

// 输出:
// Count is 1    
\end{lstlisting}
\end{columns}
\end{frame}

\begin{frame}[fragile]{注意事项}
  \begin{itemize}
      \item \Once 确保的目标操作是通过自己的\texttt{Do()}函数注册的\alert{第一个}回调函数
      \item \Once 内部使用锁,在回调函数调用前请求锁,而在调用返回前释放锁,需要留意不同回调函数之间的依赖性,避免死锁
  \end{itemize}  

\begin{lstlisting}[caption={\Once 的使用不当导致死锁}]
var onceA, onceB sync.Once
var initB func()

initA := func() { onceB. Do(initB) }
initB = func() { onceA. Do(initA) }

onceA. Do(initA)    

// 输出:
// fatal error: all goroutines are asleep - deadlock!
\end{lstlisting}
\end{frame}
    %\subsection{池\Pool }
\begin{frame}{池\Pool }
    \begin{itemize}
        \item \textbf{用途}: \Pool 创建并导出特定数目或一批资源以被第三方使用
        \item 3个基本方法
            \begin{description}
                \item[New] 创建资源实例
                \item[Get] 调用时,如果池有可用的实例,则将其直接返回给请求方,否则调用\code{New}方法实时创建一个,然后将其返回给请求方
                \item[Put] 将服务完毕的实例放回池缓存以被其他进程使用
            \end{description}
    \end{itemize}
\end{frame}

\begin{frame}[fragile]{实例}
    利用池创建对象,\alert{使用完毕后重新放回池中},从而节省对象的创建
\begin{lstlisting}
myPool := &sync.Pool{
  New: func() interface{} {
    fmt.Println("Creating new instance.")
    return struct{}{}
  },
}
myPool.Get()
instance := myPool.Get()
myPool.Put(instance)
myPool.Get()

// 输出:
// Creating new instance.
// Creating new instance.
\end{lstlisting}

另一个例子是\mhref{https://github.com/sammyne/concurrency-in-go/blob/master/chapter03/sync.pkg/pool/basic2.go}{使用池缓存资源以节省内存空间}

\end{frame}

\begin{frame}{更多使用场景}
    预存一批能够快速响应(初始化比较漫长)的服务提供方,相关代码参见
    \begin{itemize}
        \item \mhref{https://github.com/sammyne/concurrency-in-go/blob/master/chapter03/sync.pkg/pool/slow_network_service_benchmark_test.go}{没有使用池预存对象的方式}
        \item \mhref{https://github.com/sammyne/concurrency-in-go/blob/master/chapter03/sync.pkg/pool/slow_network_service_benchmark_test.go}{使用池预存对象的方式}
    \end{itemize}

    \bigskip
    对比结果是:使用缓存池的方式快了近 1000 倍
\end{frame}

\begin{frame}{注意事项}
\begin{exampleblock}{适用条件}
    \begin{itemize}
        \item 为并发的进程提供服务对象,这些对象需要满足
            \begin{itemize}
                \item 在初始化后很快就被抛弃
                \item 或创建会对内存影响不小
            \end{itemize}
    \end{itemize}
\end{exampleblock}    

\pause
\begin{alertblock}{不适用条件}
   所需服务对象不是同态的---部分时间会被浪费在对象的类型转换上 
\end{alertblock}
\end{frame}

\begin{frame}{\Pool 小结}
    \begin{itemize}
        \item 初始化\Pool 时,创建一个\code{New}成员
        \item 不要对\code{Get()}返回的对象状态做任何前提假设
        \item 不要忘记利用\code{Put()}将服务对象再次缓存以重用
        \item 池中对象应基本同态,即类型相同    
    \end{itemize}
\end{frame}

    \iffalse
    \section{Section 1}
    \begin{frame}{Simple frame}
        This is a simple frame.
    \end{frame}

    \begin{frame}[plain]{Plain frame}
        This is a frame with plain style and it is numbered.
    \end{frame}
    
    \begin{frame}[t]
        This frame has an empty title and is aligned to top.
    \end{frame}
    
    \begin{frame}[noframenumbering]{No frame numbering}
        This frame is not numbered and is citing reference \cite{knuth74}.
    \end{frame}
    
    \begin{frame}{Typesetting and Math}
        The packages \texttt{inputenc} and \texttt{FiraSans}\footnote{\url{https://fonts.google.com/specimen/Fira+Sans}}\textsuperscript{,}\footnote{\url{http://mozilla.github.io/Fira/}} are used to properly set the main fonts.
        \vfill
        This theme provides styling commands to typeset \emph{emphasized}, \alert{alerted}, \textbf{bold}, \textcolor{example}{example text}, \dots
        \vfill
        \texttt{FiraSans} also provides support for mathematical symbols:
        \begin{equation*}
            e^{i\pi} + 1 = 0.
        \end{equation*}
    \end{frame}

    \section{Section 2}
    \begin{frame}{Blocks}
        \begin{block}{Block}
            Text.
        \end{block}
        \pause
        \begin{alertblock}{Alert block}
            Alert \alert{text}.
        \end{alertblock}
        \pause
        \begin{exampleblock}{Example block}
            Example \textcolor{example}{text}.
        \end{exampleblock}
    \end{frame}
    
    \begin{frame}{Lists}
        \begin{columns}[t, onlytextwidth]
            \column{0.33\textwidth}
                Items:
                \begin{itemize}
                    \item Item 1
                    \begin{itemize}
                        \item Subitem 1.1
                        \item Subitem 1.2
                    \end{itemize}
                    \item Item 2
                    \item Item 3
                \end{itemize}
            
            \column{0.33\textwidth}
                Enumerations:
                \begin{enumerate}
                    \item First
                    \item Second
                    \begin{enumerate}
                        \item Sub-first
                        \item Sub-second
                    \end{enumerate}
                    \item Third
                \end{enumerate}
            
            \column{0.33\textwidth}
                Descriptions:
                \begin{description}
                    \item[First] Yes.
                    \item[Second] No.
                \end{description}
        \end{columns}
    \end{frame}
\setbeamertemplate{caption}[numbered]
    \begin{frame}{Figures and Tables}
        \begin{columns}
            \column{0.4\textwidth}
                \begin{figure}
                    \centering
                    \includegraphics{focuslogo.pdf}
                    \caption{Figure caption.}
                    \label{fig:focuslogo}
                \end{figure}
                
            \column{0.6\textwidth}
                \begin{table}
                    \centering
                    \begin{tabular}{rcc}
                         & Heading 1 & Heading 2 \\\hline
                        Row 1 & \(v_{11}\) & \(v_{12}\) \\
                        Row 2 & \(v_{21}\) & \(v_{22}\) \\
                        Row 3 & \(v_{31}\) & \(v_{32}\) \\
                    \end{tabular}
                    \caption{Table caption.}
                    \label{tab:demo}
                \end{table}
        \end{columns}
    \end{frame}
    
    \begin{frame}[focus]
        Thanks for using \textbf{Focus}!
    \end{frame}
    
    \appendix
    \begin{frame}{References}
        \nocite{*}
        \bibliography{demo_bibliography}
        \bibliographystyle{plain}
    \end{frame}
    
    \begin{frame}{Backup frame}
        \usebeamercolor[fg]{normal text}
        This is a backup frame, useful to include additional material for questions from the audience.
        \vfill
        The package \texttt{appendixnumberbeamer} is used not to number appendix frames.
    \end{frame}
    \fi
\end{document}
